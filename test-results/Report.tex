% Options for packages loaded elsewhere
\PassOptionsToPackage{unicode}{hyperref}
\PassOptionsToPackage{hyphens}{url}
%
\documentclass[
]{article}
\usepackage[a4paper, margin=1in]{geometry}
\usepackage{lmodern}
\usepackage{amssymb,amsmath}
\usepackage{ifxetex,ifluatex}
\ifnum 0\ifxetex 1\fi\ifluatex 1\fi=0 % if pdftex
  \usepackage[T1]{fontenc}
  \usepackage[utf8]{inputenc}
  \usepackage{textcomp} % provide euro and other symbols
\else % if luatex or xetex
  \usepackage{unicode-math}
  \defaultfontfeatures{Scale=MatchLowercase}
  \defaultfontfeatures[\rmfamily]{Ligatures=TeX,Scale=1}
\fi
% Use upquote if available, for straight quotes in verbatim environments
\IfFileExists{upquote.sty}{\usepackage{upquote}}{}
\IfFileExists{microtype.sty}{% use microtype if available
  \usepackage[]{microtype}
  \UseMicrotypeSet[protrusion]{basicmath} % disable protrusion for tt fonts
}{}
\makeatletter
\@ifundefined{KOMAClassName}{% if non-KOMA class
  \IfFileExists{parskip.sty}{%
    \usepackage{parskip}
  }{% else
    \setlength{\parindent}{0pt}
    \setlength{\parskip}{6pt plus 2pt minus 1pt}}
}{% if KOMA class
  \KOMAoptions{parskip=half}}
\makeatother
\usepackage{xcolor}
\IfFileExists{xurl.sty}{\usepackage{xurl}}{} % add URL line breaks if available
\IfFileExists{bookmark.sty}{\usepackage{bookmark}}{\usepackage{hyperref}}
\hypersetup{
  hidelinks,
  pdfcreator={LaTeX via pandoc}}
\urlstyle{same} % disable monospaced font for URLs
\usepackage{color}
\usepackage{fancyvrb}
\newcommand{\VerbBar}{|}
\newcommand{\VERB}{\Verb[commandchars=\\\{\}]}
\DefineVerbatimEnvironment{Highlighting}{Verbatim}{commandchars=\\\{\}}
% Add ',fontsize=\small' for more characters per line
\newenvironment{Shaded}{}{}
\newcommand{\AlertTok}[1]{\textcolor[rgb]{1.00,0.00,0.00}{\textbf{#1}}}
\newcommand{\AnnotationTok}[1]{\textcolor[rgb]{0.38,0.63,0.69}{\textbf{\textit{#1}}}}
\newcommand{\AttributeTok}[1]{\textcolor[rgb]{0.49,0.56,0.16}{#1}}
\newcommand{\BaseNTok}[1]{\textcolor[rgb]{0.25,0.63,0.44}{#1}}
\newcommand{\BuiltInTok}[1]{#1}
\newcommand{\CharTok}[1]{\textcolor[rgb]{0.25,0.44,0.63}{#1}}
\newcommand{\CommentTok}[1]{\textcolor[rgb]{0.38,0.63,0.69}{\textit{#1}}}
\newcommand{\CommentVarTok}[1]{\textcolor[rgb]{0.38,0.63,0.69}{\textbf{\textit{#1}}}}
\newcommand{\ConstantTok}[1]{\textcolor[rgb]{0.53,0.00,0.00}{#1}}
\newcommand{\ControlFlowTok}[1]{\textcolor[rgb]{0.00,0.44,0.13}{\textbf{#1}}}
\newcommand{\DataTypeTok}[1]{\textcolor[rgb]{0.56,0.13,0.00}{#1}}
\newcommand{\DecValTok}[1]{\textcolor[rgb]{0.25,0.63,0.44}{#1}}
\newcommand{\DocumentationTok}[1]{\textcolor[rgb]{0.73,0.13,0.13}{\textit{#1}}}
\newcommand{\ErrorTok}[1]{\textcolor[rgb]{1.00,0.00,0.00}{\textbf{#1}}}
\newcommand{\ExtensionTok}[1]{#1}
\newcommand{\FloatTok}[1]{\textcolor[rgb]{0.25,0.63,0.44}{#1}}
\newcommand{\FunctionTok}[1]{\textcolor[rgb]{0.02,0.16,0.49}{#1}}
\newcommand{\ImportTok}[1]{#1}
\newcommand{\InformationTok}[1]{\textcolor[rgb]{0.38,0.63,0.69}{\textbf{\textit{#1}}}}
\newcommand{\KeywordTok}[1]{\textcolor[rgb]{0.00,0.44,0.13}{\textbf{#1}}}
\newcommand{\NormalTok}[1]{#1}
\newcommand{\OperatorTok}[1]{\textcolor[rgb]{0.40,0.40,0.40}{#1}}
\newcommand{\OtherTok}[1]{\textcolor[rgb]{0.00,0.44,0.13}{#1}}
\newcommand{\PreprocessorTok}[1]{\textcolor[rgb]{0.74,0.48,0.00}{#1}}
\newcommand{\RegionMarkerTok}[1]{#1}
\newcommand{\SpecialCharTok}[1]{\textcolor[rgb]{0.25,0.44,0.63}{#1}}
\newcommand{\SpecialStringTok}[1]{\textcolor[rgb]{0.73,0.40,0.53}{#1}}
\newcommand{\StringTok}[1]{\textcolor[rgb]{0.25,0.44,0.63}{#1}}
\newcommand{\VariableTok}[1]{\textcolor[rgb]{0.10,0.09,0.49}{#1}}
\newcommand{\VerbatimStringTok}[1]{\textcolor[rgb]{0.25,0.44,0.63}{#1}}
\newcommand{\WarningTok}[1]{\textcolor[rgb]{0.38,0.63,0.69}{\textbf{\textit{#1}}}}
\usepackage{longtable,booktabs}
% Correct order of tables after \paragraph or \subparagraph
\usepackage{etoolbox}
\makeatletter
\patchcmd\longtable{\par}{\if@noskipsec\mbox{}\fi\par}{}{}
\makeatother
% Allow footnotes in longtable head/foot
\IfFileExists{footnotehyper.sty}{\usepackage{footnotehyper}}{\usepackage{footnote}}
\makesavenoteenv{longtable}
\setlength{\emergencystretch}{3em} % prevent overfull lines
\providecommand{\tightlist}{%
  \setlength{\itemsep}{0pt}\setlength{\parskip}{0pt}}
\setcounter{secnumdepth}{-\maxdimen} % remove section numbering

\author{Marcin Cieślak}
\date{2024}
\title{Post-quantum cryptography on ARM Cortex-M7: a report}

\begin{document}

\hypertarget{title}{%
\part{Post-quantum cryptography on ARM Cortex M7: a report}
\label{title}}

The aim of this report is to evaluate the results of an experiment regarding implementing PQ cryptography
in an embedded system and to analyse whether it is currently feasible resource-wise to employ such mechanisms in a practical application.

\hypertarget{abstract}{%
\part{Abstract}
\label{abstract}}

With the introduction of Shor's algorithm\footnote{\url{https://en.wikipedia.org/wiki/Shor\%27s_algorithm}}
the long-term future of currently used public-key signature schemes relying on either the large integer prime factorisation problem\footnote{\url{https://en.wikipedia.org/wiki/Integer_factorization}} or
elliptic curve properties\footnote{\url{https://en.wikipedia.org/wiki/Discrete_logarithm}} is uncertain.
Although there are currently no practical devices capable of employing
Shor's algorithm to solve any of the aforementioned problems, it is probable that they may be created one day.
This in turn creates need for future-proof algorithms that are known to be resistant to Shor's algorithm shall a quantum computer powerful enough to threaten the status-quo of cryptography be ever built.

\hypertarget{urgency}{%
\section{Urgency of the matter}\label{urgency}}

Even though one may feel tempted to assume that, given that the threat from quantum computers is distant,
there is currently no need to bother. While such an assumption is not necessarily wrong for most
industrial and consumer market applications, state-level secrets and military applications might require
protection from decryption of past cipher-texts in the future\footnote{\url{https://en.wikipedia.org/wiki/Harvest_now,_decrypt_later}}.
What is more, this future may actually be less distant than it seems\footnote{\url{https://semiengineering.com/the-race-toward-quantum-advantage}}.

\hypertarget{possible-resolutions}{%
\section{Solutions at hand}\label{possible-resolutions}}

Even though practical quantum computers are yet to arrive, cryptography resistant to Shor's algorithm is already out in the wild. A performance and key size penalty relative to ECDSA is
present\footnote{\url{https://blog.cloudflare.com/nist-post-quantum-surprise\#digital-signatures}},
although it is manageable especially for algorithms based on latices, specifically pqNTRUSign, CRYSTALS-Dilithium\footnote{\url{https://www.pq-crystals.org/dilithium/index.shtml}} and Falcon\footnote{\url{https://falcon-sign.info/}}.
Out of them, Dilithium and Falcon where selected by NIST\footnote{\url{https://en.wikipedia.org/wiki/NIST_Post-Quantum_Cryptography_Standardization\#Selected_Algorithms_2022}} as early variants of NTRUSign
had serious security flaws\footnote{\url{https://en.wikipedia.org/wiki/NTRUSign\#Security}} which brought
attention to how tricky it is to properly choose NTRUSign's parameters and how serious the consequences
of bad choices in that field to the overall security of the system might be (private key recovery was possible with as low as 400 signatures).
While it may seem trivial, having a degree of resistance to common mistakes is an important part of a reliable cryptographic system as they are bound to happen (and e.g. ED25519 is designed with robust
safeguards against broken PRNG's in mind\footnote{\url{https://en.wikipedia.org/wiki/EdDSA\#Secure_coding}}).

\hypertarget{embedded-options}{%
\section{Options for the embedded world}\label{embedded-options}}

The strongest constrain in the world of embedded devices is memory. Operations with high computational
cost can be offloaded to special-case peripheral modules (which will arise sooner or later as it is
currently the case for AES and SHA even in lower-end micro-controllers) or run over time between other
tasks if possible. On the other hand, expanding memory creates additional costs, complexity, reduces
performance and introduces additional failure points.

Due to the above factors CRYSTALS-Dilithium and SPHINCS+, although certainly possible to implement even
for the slightly smaller Cortex M4 cores\footnote{\url{https://github.com/mupq/pqm4}}, are in an
extremely disadvantageous position owing to the respectively large and gargantuan amounts of memory
required to hold just the computed signature. The author of this document acknowledges that CRYSTALS-Dilithium is the one of the fastest algorithms and that SPHINCS+ provides methods which allow to avoid holding the whole signature in memory at once. Regardless of that, algorithms which have no need for workarounds to this problem are more appealing and were therefore considered in the experiment.

\hypertarget{the-experiment}{%
\part{The experiment}
\label{the-experiment}}

The experiment involved porting pqNTRUSign and Falcon to the micro-controller from NIST submission packages, to verify if the code operates properly and to judge the code's overall quality in terms of readability, documentation, memory safety and following general programming best practices.

The second part was to design a simple communication scheme based on \textit{AT} commands (normally used for controlling LTE and Bluetooth communication modules) so as to allow transmitting messages and receiving signatures as hex-encoded strings (for simplicity). The communication scheme also includes miscellaneous commands used for reading the keys and running benchmarks.

The third and final part was comparing performance and memory usage (also by each software component).

\hypertarget{at-commands}{%
\section{Command description}\label{at-commands}}

The communication scheme designed was composed of the following commands:

\begin{longtable}[]{@{}llll@{}}
\toprule
Command & Arguments & Purpose\tabularnewline
\midrule
\endhead
\textit{AT+S} & \textit{<hex\_string>} &
signs the \textit{<hex\_string>} and returns the signed data as \textit{<hex\_string>}
\tabularnewline
\textit{AT+V} & \textit{<hex\_string>} &
verifies the \textit{<hex\_string>} signed message
\tabularnewline
\textit{AT+I} & &
outputs maximum message size and currently used algorithm
\tabularnewline
\textit{AT+G} & &
generates a new key pair
\tabularnewline
\textit{AT+P} & &
returns public key as \textit{<hex\_string>}
\tabularnewline
\textit{AT+K} & &
returns private key as \textit{<hex\_string>}
\tabularnewline
\textit{AT+R} & &
runs keygen benchmark and returns average time per keygen in \textit{ms}
\tabularnewline
\textit{AT+B} & &
runs signing benchmark and returns average time per signature in \textit{ms}
\tabularnewline
\bottomrule
\end{longtable}

\hypertarget{port-results}{%
\section{Results of the porting process}\label{port-results}}

The author was successful in porting both the libraries onto the target architecture (Cortex M7). Falcon was a lot easier to do since it included a variant targeting Cortex M specifically and only a
reallocation of the temporary buffers to outer memory regions was required as the particular chip used featured multiple memory banks. pqNTRUSign was harder yet manageable. The hardware RNG module equipped in the chip was also successfully employed.

\hypertarget{code-quality}{%
\section{Code quality opinions}\label{code-quality}}

While the author of this document is not competent enough to judge the quality of implemented mathematical functions, it was nonetheless possible to notice that pqNTRUSign code was not free from
common programming mistakes like returning pointers to local variables, omitting \textit{const} from large read-only chunks of data, boilerplate pointer manipulation or, the worst of all, accepting a returned \textit{NULL} pointer without any safeguards and happily writing some 32kB there.
This particular issue has been reported to the maintainer\footnote{\url{https://github.com/zhenfeizhang/pqNTRUSign/pull/2}}.
Comments were also hard to come by and a standard library function was used for getting a random number once. All these factors altogether make one feel uncertain whether the original code can be trusted.

On the other hand, Falcon mostly just worked\textsuperscript{TM}. There were no compiler warnings and the API to implement requesting random bytes from hardware RNG was obvious. The code was well documented and programmed in a consistent style which made one felt assured that the designer knew what he was doing.

\hypertarget{benchmarks-all}{%
\part{Benchmarks and memory analysis}\label{benchmarks-all}}

\hypertarget{benchmarks-for-pqntrusign}{%
\section{Benchmarks for pqNTRUSign}\label{benchmarks-for-pqntrusign}}

\hypertarget{setup}{%
\subsection{Setup}\label{setup}}

\begin{itemize}
\tightlist
\item
  \textbf{Device:} NUCLEO-H753ZI
\item
  \textbf{Compiler:} arm-none-eabi-gcc 10.3.1
\item
  \textbf{Public key size:} 2065
\item
  \textbf{Private key size:} 2604
\item
  \textbf{AES equivalent:} 128 \textit{(NIST 1)}
\end{itemize}

Lower key sizes will not be tested as the cryptography already
has security concerns.
See relevant Wikipedia article on
\href{https://en.wikipedia.org/wiki/NTRUSign}{NTRUSign}.

To communicate with the device, a serial terminal is required. On Linux
the device is usually located at \texttt{/dev/ttyACMx}. See
\texttt{journalctl\ -efn10} for the exact device name.

Example command to start a serial terminal:

\begin{Shaded}
\begin{Highlighting}[]
\ExtensionTok{picocom}\NormalTok{ {-}{-}omap crcrlf {-}{-}echo {-}b 115200 /dev/ttyACM0}
\end{Highlighting}
\end{Shaded}

All the \texttt{AT+x} commands are expected to be terminated with a
newline or carret return \textbf{and} newline.

\hypertarget{key-generation-over-25-runs}{%
\subsection{Key generation over 25
runs}\label{key-generation-over-25-runs}}

\begin{longtable}[]{@{}ll@{}}
\toprule
Compiler Preset & Time (ms)\tabularnewline
\midrule
\endhead
Release & 818\tabularnewline
MinSizeRel & 875\tabularnewline
Debug & 1465\tabularnewline
\bottomrule
\end{longtable}

\hypertarget{running-the-keygen-benchmark}{%
\subsubsection{Running the keygen
benchmark}\label{running-the-keygen-benchmark}}

In a serial terminal, run the following command:

\texttt{AT+R}

\hypertarget{signature-generation-over-25-runs}{%
\subsection{Signature generation over 25
runs}\label{signature-generation-over-25-runs}}

\begin{longtable}[]{@{}lll@{}}
\toprule
Variant & Compiler Preset & Time (ms)\tabularnewline
\midrule
\endhead
Gaussian distr. & Debug & 7596\tabularnewline
Gaussian distr. & MinSizeRel & 2029\tabularnewline
Gaussian distr. & Release & 1598\tabularnewline
Uniform distr. & Debug & 2749\tabularnewline
Uniform distr. & MinSizeRel & 627\tabularnewline
Uniform distr. & Release & 522\tabularnewline
\bottomrule
\end{longtable}

\hypertarget{running-the-signature-benchmark}{%
\subsubsection{Running the signature
benchmark}\label{running-the-signature-benchmark}}

In a serial terminal, run the following command:

\texttt{AT+B}

\hypertarget{memory-footprint}{%
\subsection{Memory footprint}\label{memory-footprint}}

\hypertarget{overall-usage}{%
\subsubsection{Overall usage}\label{overall-usage}}

The memory footprint is calculated by substracting the ammount of memory
used by a blank application from the memory used by the application with
the cryptographic library.

\begin{longtable}[]{@{}llllll@{}}
\toprule
Compiler Preset & Variant & Flash (bytes) & DTCRAM (bytes) & BUF RAM
(bytes) & Total RAM (bytes)\tabularnewline
\midrule
\endhead
Release & Gaussian distr. & 73696 & 8344 & 225856 &
234200\tabularnewline
MinSizeRel & Gaussian distr. & 68316 & 8344 & 225853 &
234197\tabularnewline
Release & Uniform distr. & 73340 & 8344 & 225856 & 234200\tabularnewline
MinSizeRel & Uniform distr. & 68204 & 8344 & 225853 &
234197\tabularnewline
\bottomrule
\end{longtable}

As one may see, the differences between the variants are not significant
and the MinSizeRel preset obviously uses less FLASH memory. On average
\textasciitilde235kB of RAM and \textasciitilde70kB of FLASH memory is
used by the cryptographic library.

\hypertarget{detailed-flash-usage-per-function}{%
\subsubsection{Detailed FLASH usage per
function}\label{detailed-flash-usage-per-function}}

\begin{longtable}[]{@{}llll@{}}
\toprule
File & Function & FLASH (bytes: hex) & FLASH (kB)\tabularnewline
\midrule
\endhead
crypto\_hash\_sha512.c & crypto\_hashblocks\_sha512 & 00007c80 &
000031.5\tabularnewline
crypto\_hash\_sha512.c & crypto\_hash\_sha512 & 00000158 &
000000.3\tabularnewline
crypto\_sream\_salsa20.c & crypto\_core\_salsa20 & 0000038c &
000000.9\tabularnewline
crypto\_sream\_salsa20.c & crypto\_sream & 000000ec &
000000.2\tabularnewline
DGS.c.obj & DGS & 00000188 & 000000.4\tabularnewline
DGS.c.obj & DDGS & 00000110 & 000000.3\tabularnewline
fastrandombytes.c.obj & fastrandombytes & 0000007c &
000000.2\tabularnewline
fastrandombytes.c.obj & rng\_cleanup & 00000040 &
000000.1\tabularnewline
fastrandombytes.c.obj & rng\_init & 00000088 & 000000.2\tabularnewline
fastrandombytes.c.obj & rng\_uint16 & 000000b0 & 000000.2\tabularnewline
fastrandombytes.c.obj & rng\_uint64 & 0000012c & 000000.3\tabularnewline
fastrandombytes.c.obj & randpool & 00001000 & 000004.0\tabularnewline
fastrandombytes.c.obj & randpos & 00000002 & 000000.0\tabularnewline
misc.c.obj & max\_norm & 00000040 & 000000.1\tabularnewline
misc.c.obj & get\_scala & 00000050 & 000000.1\tabularnewline
NTT.c.obj & extendedEuclid & 000000f0 & 000000.2\tabularnewline
NTT.c.obj & NTT & 000001bc & 000000.4\tabularnewline
NTT.c.obj & Inv\_NTT & 00000150 & 000000.3\tabularnewline
NTT.c.obj & InvMod & 0000017c & 000000.3\tabularnewline
packing.c.obj & tri\_to\_string & 0000016c & 000000.3\tabularnewline
packing.c.obj & string\_to\_tri & 00000290 & 000000.6\tabularnewline
packing.c.obj & pack\_public\_key & 0000008c & 000000.2\tabularnewline
packing.c.obj & unpack\_public\_key & 00000064 & 000000.1\tabularnewline
packing.c.obj & pack\_secret\_key & 000001fc & 000000.5\tabularnewline
packing.c.obj & unpack\_secret\_key & 0000020c & 000000.5\tabularnewline
param.c.obj & pq\_get\_param\_set\_by\_id & 0000001c &
000000.0\tabularnewline
poly.c.obj & karatsuba & 00000288 & 000000.7\tabularnewline
poly.c.obj & is\_balance & 0000001c & 000000.0\tabularnewline
poly.c.obj & pol\_gen\_flat & 000004b8 & 000001.2\tabularnewline
poly.c.obj & binary\_poly\_gen & 00000214 & 000000.5\tabularnewline
poly.c.obj & pol\_unidrnd\_pZ & 000000b0 & 000000.2\tabularnewline
poly.c.obj & pol\_unidrnd & 00000090 & 000000.1\tabularnewline
poly.c.obj & pol\_unidrnd\_with\_seed & 000000cc &
000000.3\tabularnewline
poly.c.obj & pol\_inv\_mod2 & 00000330 & 000000.8\tabularnewline
poly.c.obj & cmod & 0000004c & 000000.1\tabularnewline
poly.c.obj & pol\_mul\_coefficients & 000000c4 & 000000.2\tabularnewline
poly.c.obj & pol\_mul\_mod\_p & 000000d0 & 000000.2\tabularnewline
pqNTRUSign.c.obj & normf & 00000108 & 000000.3\tabularnewline
pqNTRUSign.c.obj & keygen & 00000148 & 000000.3\tabularnewline
pqNTRUSign.c.obj & rejection\_sampling & 000000ec &
000000.2\tabularnewline
pqNTRUSign.c.obj & challenge & 000000f4 & 000000.2\tabularnewline
pqNTRUSign.c.obj & sign & 000002bc & 000000.7\tabularnewline
pqNTRUSign.c.obj & verify & 00000130 & 000000.3\tabularnewline
rng.c.obj & \_32\_randombytes & 00000040 & 000000.1\tabularnewline
rng.c.obj & rng\_init & 00000088 & 000000.2\tabularnewline
rng.c.obj & rng\_uint16 & 000000b0 & 000000.2\tabularnewline
rng.c.obj & rng\_uint64 & 0000012c & 000000.3\tabularnewline
\bottomrule
\end{longtable}

\hypertarget{per-component-sums}{%
\paragraph{Per component sums}\label{per-component-sums}}

\begin{longtable}[]{@{}ll@{}}
\toprule
File & FLASH (kB)\tabularnewline
\midrule
\endhead
crypto\_hash\_sha512.c & 000031.8\tabularnewline
crypto\_sream\_salsa20.c & 000001.1\tabularnewline
DGS.c.obj & 000000.7\tabularnewline
fastrandombytes.c.obj & 000004.5\tabularnewline
misc.c.obj & 000000.2\tabularnewline
NTT.c.obj & 000000.9\tabularnewline
packing.c.obj & 000001.2\tabularnewline
param.c.obj & 000000.0\tabularnewline
poly.c.obj & 000004.6\tabularnewline
pqNTRUSign.c.obj & 000001.5\tabularnewline
rng.c.obj & 000000.6\tabularnewline
\textbf{TOTAL} & 000047.1\tabularnewline
\bottomrule
\end{longtable}

As one can easily see, the SHA512 implementation takes the most FLASH
memory. Using the hardware hash module would be a good idea to reduce
the memory footprint.

The resulting size would be arround 15kB for functions + 16kB of
constants defined in the \texttt{param.c} file.

\hypertarget{script-used}{%
\paragraph{Script used}\label{script-used}}

\begin{Shaded}
\begin{Highlighting}[]
\KeywordTok{for} \FunctionTok{file}\NormalTok{ in *.obj}\KeywordTok{;}
\KeywordTok{do} \BuiltInTok{echo} \VariableTok{$file}\KeywordTok{;}
\ExtensionTok{arm{-}none{-}eabi{-}objdump}\NormalTok{ {-}h }\VariableTok{$file}  \KeywordTok{|} \FunctionTok{tail}\NormalTok{ +6 }\KeywordTok{|} \FunctionTok{grep}\NormalTok{ {-}E }\StringTok{"^{}.*[0{-}9]"} \KeywordTok{|} \textbackslash
\FunctionTok{awk} \StringTok{\textquotesingle{}\{print $2 " " $3\}\textquotesingle{}}\KeywordTok{;} \KeywordTok{done} \KeywordTok{|} \ExtensionTok{column}\NormalTok{ {-}t }
\end{Highlighting}
\end{Shaded}

\hypertarget{other-findings}{%
\subsection{Other findings}\label{other-findings}}

\hypertarget{broken-signatures}{%
\subsubsection{Broken signatures}\label{broken-signatures}}

An interesting observation is that for some combinations of generated
keys and messages, the signature generation may result in a signature
that is not valid. This is circumvented by checking the signature before
sending it back to the caller. It is therefore possible to receive an
error response even if correct parameters were passed. Researching the
reason behind this behaviour is out of the scope of this project as a
workaround is already in place.

\hypertarget{lack-of-memory-safety-in-crypto_sign_open}{%
\subsubsection{\texorpdfstring{Lack of memory safety in
\texttt{crypto\_sign\_open()}}{Lack of memory safety in crypto\_sign\_open()}}\label{lack-of-memory-safety-in-crypto_sign_open}}

So apparently the \texttt{crypto\_sign\_open()} function will cause out
of bounds memory access if the supplied signature is misaligned. This
issue has been fixed in \texttt{unpack\_public\_key()} by adding a check
if the returned pointer is not \texttt{NULL}. I hope there are no more
eastereggs like this one (it bricked the device until a full reflashing
was performed).

\hypertarget{benchmarks-for-falcon-512}{%
\section{Benchmarks for Falcon-512}\label{benchmarks-for-falcon-512}}

\hypertarget{setup-1}{%
\subsection{Setup}\label{setup-1}}

\begin{itemize}
\tightlist
\item
  \textbf{Device:} NUCLEO-H753ZI
\item
  \textbf{Compiler:} arm-none-eabi-gcc 10.3.1
\item
  \textbf{Public key size:} 897
\item
  \textbf{Private key size:} 1281
\item
  \textbf{AES equivalent:} 128 \textit{(NIST 1)}
\end{itemize}

Communication as in \url{pqNTRUSign.md}.

\hypertarget{key-generation-over-50-runs}{%
\subsection{Key generation over 50
runs}\label{key-generation-over-50-runs}}

\begin{longtable}[]{@{}ll@{}}
\toprule
Compiler Preset & Time (ms)\tabularnewline
\midrule
\endhead
Release & 266\tabularnewline
MinSizeRel & 308\tabularnewline
Debug & 808\tabularnewline
\bottomrule
\end{longtable}

\hypertarget{signature-generation-over-50-runs}{%
\subsection{Signature generation over 50
runs}\label{signature-generation-over-50-runs}}

\begin{longtable}[]{@{}ll@{}}
\toprule
Compiler Preset & Time (ms)\tabularnewline
\midrule
\endhead
Release & 61\tabularnewline
MinSizeRel & 61\tabularnewline
Debug & 101\tabularnewline
\bottomrule
\end{longtable}

\hypertarget{memory-footprint-1}{%
\subsection{Memory footprint}\label{memory-footprint-1}}

\hypertarget{overall-usage-1}{%
\subsubsection{Overall usage}\label{overall-usage-1}}

\begin{longtable}[]{@{}lll@{}}
\toprule
Compiler Preset & FLASH (bytes) & Total RAM (bytes)\tabularnewline
\midrule
\endhead
Release & 76292 & 64514\tabularnewline
MinSizeRel & 60212 & 64514\tabularnewline
Debug & 82048 & 64538\tabularnewline
\bottomrule
\end{longtable}

\hypertarget{size-of-individual-symbols-in-object-files-release}{%
\subsubsection{Size of individual symbols in object files
(Release)}\label{size-of-individual-symbols-in-object-files-release}}

\begin{longtable}[]{@{}llll@{}}
\toprule
File & Symbol & FLASH (bytes: hex) & FLASH (kB)\tabularnewline
\midrule
\endhead
api.c.obj & crypto\_sign\_keypair & 000000f0 & 0.23\tabularnewline
api.c.obj & crypto\_sign & 000001f0 & 0.48\tabularnewline
api.c.obj & crypto\_sign\_open & 000000e4 & 0.22\tabularnewline
codec.c.obj & falcon\_inner\_modq\_encode & 00000094 &
0.15\tabularnewline
codec.c.obj & falcon\_inner\_modq\_decode & 00000070 &
0.11\tabularnewline
codec.c.obj & falcon\_inner\_trim\_i8\_encode & 000000b8 &
0.18\tabularnewline
codec.c.obj & falcon\_inner\_trim\_i8\_decode & 000000b4 &
0.18\tabularnewline
codec.c.obj & falcon\_inner\_comp\_encode & 000000c8 &
0.19\tabularnewline
codec.c.obj & falcon\_inner\_comp\_decode & 0000009c &
0.16\tabularnewline
codec.c.obj & falcon\_inner\_max\_FG\_bits & 0000000b &
0.01\tabularnewline
codec.c.obj & falcon\_inner\_max\_fg\_bits & 0000000b &
0.01\tabularnewline
common.c.obj & falcon\_inner\_hash\_to\_point\_vartime & 00000074 &
0.12\tabularnewline
common.c.obj & falcon\_inner\_is\_short & 00000050 & 0.08\tabularnewline
common.c.obj & falcon\_inner\_is\_short\_half & 00000048 &
0.07\tabularnewline
common.c.obj & l2bound & 0000002c & 0.04\tabularnewline
common.c.obj & overtab.0 & 00000016 & 0.03\tabularnewline
fft.c.obj & falcon\_inner\_FFT & 000001d0 & 0.45\tabularnewline
fft.c.obj & falcon\_inner\_iFFT & 0000021c & 0.53\tabularnewline
fft.c.obj & falcon\_inner\_poly\_add & 0000002c & 0.04\tabularnewline
fft.c.obj & falcon\_inner\_poly\_sub & 00000030 & 0.05\tabularnewline
fft.c.obj & falcon\_inner\_poly\_neg & 00000020 & 0.03\tabularnewline
fft.c.obj & falcon\_inner\_poly\_adj\_fft & 00000034 &
0.05\tabularnewline
fft.c.obj & falcon\_inner\_poly\_mul\_fft & 000000c8 &
0.20\tabularnewline
fft.c.obj & falcon\_inner\_poly\_muladj\_fft & 000000d4 &
0.21\tabularnewline
fft.c.obj & falcon\_inner\_poly\_mulselfadj\_fft & 00000078 &
0.12\tabularnewline
fft.c.obj & falcon\_inner\_poly\_mulconst & 00000030 &
0.05\tabularnewline
fft.c.obj & falcon\_inner\_poly\_invnorm2\_fft & 000000ec &
0.23\tabularnewline
fft.c.obj & falcon\_inner\_poly\_add\_muladj\_fft & 000001a8 &
0.41\tabularnewline
fft.c.obj & falcon\_inner\_poly\_mul\_autoadj\_fft & 00000050 &
0.08\tabularnewline
fft.c.obj & falcon\_inner\_poly\_div\_autoadj\_fft & 0000006c &
0.11\tabularnewline
fft.c.obj & falcon\_inner\_poly\_LDL\_fft & 0000020c &
0.51\tabularnewline
fft.c.obj & falcon\_inner\_poly\_split\_fft & 000001e0 &
0.47\tabularnewline
fft.c.obj & falcon\_inner\_poly\_merge\_fft & 00000174 &
0.37\tabularnewline
fpr.c.obj & falcon\_inner\_fpr\_scaled & 00000084 & 0.13\tabularnewline
fpr.c.obj & falcon\_inner\_fpr\_add & 0000013c & 0.31\tabularnewline
fpr.c.obj & falcon\_inner\_fpr\_mul & 000000e8 & 0.23\tabularnewline
fpr.c.obj & falcon\_inner\_fpr\_div & 00000640 & 1.56\tabularnewline
fpr.c.obj & falcon\_inner\_fpr\_sqrt & 00000748 & 1.82\tabularnewline
fpr.c.obj & falcon\_inner\_fpr\_expm\_p63 & 000004e8 &
1.23\tabularnewline
fpr.c.obj & falcon\_inner\_fpr\_gm\_tab & 00004000 &
16.00\tabularnewline
fpr.c.obj & falcon\_inner\_fpr\_p2\_tab & 00000058 & 0.09\tabularnewline
keygen.c.obj & modp\_R2 & 000000d4 & 0.21\tabularnewline
keygen.c.obj & modp\_mkgm2 & 0000034c & 0.82\tabularnewline
keygen.c.obj & poly\_sub\_scaled & 00000144 & 0.33\tabularnewline
keygen.c.obj & poly\_small\_mkgauss & 00000110 & 0.26\tabularnewline
keygen.c.obj & poly\_big\_to\_fp & 000000f8 & 0.25\tabularnewline
keygen.c.obj & modp\_iNTT2\_ext.part.0 & 0000021c & 0.53\tabularnewline
keygen.c.obj & modp\_NTT2\_ext.part.0.constprop.0 & 000000e0 &
0.22\tabularnewline
keygen.c.obj & modp\_iNTT2\_ext.part.0.constprop.0 & 0000013c &
0.31\tabularnewline
keygen.c.obj & modp\_iNTT2\_ext.part.0.constprop.1 & 00000130 &
0.30\tabularnewline
keygen.c.obj & poly\_big\_to\_fp.constprop.0 & 00000080 &
0.13\tabularnewline
keygen.c.obj & poly\_big\_to\_fp.constprop.1 & 000000b4 &
0.18\tabularnewline
keygen.c.obj & zint\_rebuild\_CRT.constprop.3 & 00000260 &
0.60\tabularnewline
keygen.c.obj & poly\_sub\_scaled\_ntt & 00000498 & 1.15\tabularnewline
keygen.c.obj & zint\_co\_reduce\_mod & 00000228 & 0.54\tabularnewline
keygen.c.obj & make\_fg & 00000a40 & 2.56\tabularnewline
keygen.c.obj & solve\_NTRU\_intermediate & 00000f24 &
3.79\tabularnewline
keygen.c.obj & falcon\_inner\_keygen & 00002d1c & 11.28\tabularnewline
keygen.c.obj & BITLENGTH & 00000058 & 0.09\tabularnewline
keygen.c.obj & MAX\_BL\_LARGE & 00000028 & 0.04\tabularnewline
keygen.c.obj & MAX\_BL\_SMALL & 0000002c & 0.05\tabularnewline
keygen.c.obj & PRIMES & 00001878 & 6.12\tabularnewline
keygen.c.obj & REV10 & 00000800 & 2.00\tabularnewline
keygen.c.obj & gauss\_1024\_12289 & 000000d8 & 0.21\tabularnewline
rng.c.obj & falcon\_inner\_prng\_refill & 00000344 & 0.82\tabularnewline
rng.c.obj & falcon\_inner\_prng\_init & 0000001c & 0.03\tabularnewline
shake.c.obj & process\_block & 000008b4 & 2.18\tabularnewline
shake.c.obj & falcon\_inner\_i\_shake256\_init & 00000018 &
0.03\tabularnewline
shake.c.obj & falcon\_inner\_i\_shake256\_inject & 000000d0 &
0.20\tabularnewline
shake.c.obj & falcon\_inner\_i\_shake256\_flip & 00000030 &
0.05\tabularnewline
shake.c.obj & falcon\_inner\_i\_shake256\_extract & 00000060 &
0.10\tabularnewline
sign.c.obj & falcon\_inner\_gaussian0\_sampler & 000000c0 &
0.19\tabularnewline
sign.c.obj & falcon\_inner\_sampler & 000002d0 & 0.70\tabularnewline
sign.c.obj & ffSampling\_fft\_dyntree.constprop.0 & 000001f0 &
0.48\tabularnewline
sign.c.obj & falcon\_inner\_sign\_dyn & 00000770 & 1.86\tabularnewline
sign.c.obj & dist.0 & 000000d8 & 0.22\tabularnewline
sign.c.obj & fpr\_inv\_sigma & 00000058 & 0.09\tabularnewline
sign.c.obj & fpr\_sigma\_min & 00000058 & 0.09\tabularnewline
vrfy.c.obj & mq\_NTT & 000000d4 & 0.21\tabularnewline
vrfy.c.obj & mq\_iNTT & 00000138 & 0.30\tabularnewline
vrfy.c.obj & falcon\_inner\_to\_ntt\_monty & 00000054 &
0.09\tabularnewline
vrfy.c.obj & falcon\_inner\_verify\_raw & 00000100 & 0.25\tabularnewline
vrfy.c.obj & falcon\_inner\_compute\_public & 00000364 &
0.85\tabularnewline
vrfy.c.obj & falcon\_inner\_complete\_private & 00000458 &
1.09\tabularnewline
vrfy.c.obj & GMb & 00000800 & 2.00\tabularnewline
vrfy.c.obj & iGMb & 00000800 & 2.00\tabularnewline
\textbf{TOTAL} & & & \textbf{73.34}\tabularnewline
\bottomrule
\end{longtable}

\hypertarget{per-component-sums-release}{%
\subsubsection{Per component sums
(Release)}\label{per-component-sums-release}}

\begin{longtable}[]{@{}ll@{}}
\toprule
File & FLASH (kB)\tabularnewline
\midrule
\endhead
api.c.obj & 0.93\tabularnewline
codec.c.obj & 0.99\tabularnewline
common.c.obj & 0.34\tabularnewline
fft.c.obj & 3.91\tabularnewline
fpr.c.obj & 21.37\tabularnewline
keygen.c.obj & 31.97\tabularnewline
rng.c.obj & 0.85\tabularnewline
shake.c.obj & 2.56\tabularnewline
sign.c.obj & 3.63\tabularnewline
vrfy.c.obj & 6.79\tabularnewline
\textbf{TOTAL} & \textbf{73.34}\tabularnewline
\bottomrule
\end{longtable}

\hypertarget{benchmarks-for-falcon-1024}{%
\section{Benchmarks for Falcon-1024}\label{benchmarks-for-falcon-1024}}

\hypertarget{setup-2}{%
\subsection{Setup}\label{setup-2}}

\begin{itemize}
\tightlist
\item
  \textbf{Device:} NUCLEO-H753ZI
\item
  \textbf{Compiler:} arm-none-eabi-gcc 10.3.1
\item
  \textbf{Public key size:} 1793
\item
  \textbf{Private key size:} 2305
\item
  \textbf{AES equivalent:} 256 \textit{(NIST 5)}
\end{itemize}

Communication as in \url{pqNTRUSign.md}.

\hypertarget{key-generation-over-50-runs-1}{%
\subsection{Key generation over 50
runs}\label{key-generation-over-50-runs-1}}

\begin{longtable}[]{@{}ll@{}}
\toprule
Compiler Preset & Time (ms)\tabularnewline
\midrule
\endhead
Release & 711\tabularnewline
MinSizeRel & 853\tabularnewline
Debug & 2504\tabularnewline
\bottomrule
\end{longtable}

\hypertarget{signature-generation-over-50-runs-1}{%
\subsection{Signature generation over 50
runs}\label{signature-generation-over-50-runs-1}}

\begin{longtable}[]{@{}ll@{}}
\toprule
Compiler Preset & Time (ms)\tabularnewline
\midrule
\endhead
Release & 134\tabularnewline
MinSizeRel & 135\tabularnewline
Debug & 218\tabularnewline
\bottomrule
\end{longtable}

\hypertarget{memory-footprint-2}{%
\subsection{Memory footprint}\label{memory-footprint-2}}

\hypertarget{overall-usage-2}{%
\subsubsection{Overall usage}\label{overall-usage-2}}

\begin{longtable}[]{@{}lll@{}}
\toprule
Compiler Preset & Flash (bytes) & Total RAM (bytes)\tabularnewline
\midrule
\endhead
Release & 76400 & 128008\tabularnewline
MinSizeRel & 60280 & 128002\tabularnewline
Debug & 82132 & 128005\tabularnewline
\bottomrule
\end{longtable}

As the memory usage pattern is expected to be similar to
\href{falcon-512.md}{falcon-512}, \texttt{.obj} file analysis is omitted
(buffer sizes will change and that will mostly be it).

\hypertarget{end}{%
\part{Closing thoughts}
\label{end}}

\hypertarget{pq-on-embedded}{%
\section{Is PQ cryptography practical on embedded systems}\label{pq-on-embedded}}

It certainly is. Falcon-512 was capable of creating a signature in as few as 61 milliseconds which is a delay completely unnoticeable for an average human. It was possible to sign messages as large as 4kB while using a wasteful hex encoding method and with less than 30\% of RAM and 5\% of program memory used for the application together with the cryptography. Although quite a large micro-controller was used, the code could easily fit into a two times smaller one.

\hypertarget{winner}{%
\section{Which of the algorithms tested performed better}\label{winner}}

Falcon. With twice the security it was still ten times faster and required half the RAM with a similar program memory usage even though some 20kB where sacrificed to implement floating-point arithmetic in a known-secure way. What is more, the reference code seemed more well-maintained.

\end{document}
